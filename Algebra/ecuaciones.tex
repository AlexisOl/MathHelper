\documentclass[letterpaper]{article}
\usepackage[utf8]{inputenc}
\usepackage{amsmath} 
\usepackage{graphicx}
\usepackage{subcaption} %
\usepackage{amsfonts}
\usepackage{amssymb} 
\usepackage[spanish]{babel}
\usepackage{mathtools}

\usepackage[
  letterpaper,
  left=1cm,
  right=1cm,
  top=1.5cm,
  bottom=1.5cm
]{geometry}
\usepackage[
  final,
  unicode,
  colorlinks=true,
  citecolor=blue,
  linkcolor=blue,
  plainpages=false,
  urlcolor=blue,
  pdfpagelabels=true,
  pdfsubject={},
  pdfauthor={Alexxus},
  pdftitle={},
  pdfkeywords={}
]{hyperref}

\usepackage{lastpage}
\usepackage{fancyhdr}
\fancyhf{}
\pagestyle{fancy}
\fancyhf{}



\renewcommand{\headrulewidth}{2pt} 
\renewcommand{\footrulewidth}{2pt}

\usepackage{dirtytalk}

\title{Ecuaciones}
\author{}
\usepackage{datetime}
\date{\displaydate{date}}


\usepackage{csquotes}

\begin{document}

\maketitle
\thispagestyle{fancy}
\section{Ecuacion utilizada}

\begin{enumerate}
    \item $$2x-3y =0 $$
    \item  $$13x+y = 3$$
\end{enumerate}
    \def\proof{\paragraph{Demostración: \\}}
    \def\endproof{\hfill$\blacksquare$}
    \begin{proof}
    
        Inverso Aditivo y asociatividad$$2x-3y+3y = 3y$$
        $$2x = 3y$$
        Inverso multiplicativo del numero natural que acompania a y. $$2x * \frac{1}{3} = 3y * \frac{1}{3}$$
        $$\frac{2x}{3} = y$$
        Sustitucion en la segunda ecuacion $$13x + \frac{2x}{3}  = 3 $$ 
        Factor comun$$x(13+\frac{2}{3})=3$$ 
        Inverso aditivo $$\therefore x = \frac{3*3}{41}$$
        Asi que: $$X = \frac{9}{41}$$
        y sustituimos este valor a la ecuacion 1. $$\frac{2}{3}*\frac{9}{41} = y$$
        $$\therefore y =\frac{6}{41}$$
    \end{proof}






\end{document}